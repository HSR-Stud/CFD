




\documentclass[a4paper]{article}

% Mathematik-Pakete
\usepackage{amsmath}
\usepackage{amssymb}
\usepackage{amstext}
\usepackage{amsfonts}
\usepackage{mathrsfs}
\usepackage{paralist}


%\usepackage[ansinew]{inputenc}
%\usepackage[T1]{fontenc}

\usepackage[utf8]{inputenc}
%\usepackage[latin1]{inputenc}  
%\usepackage[T1]{fontenc}
%\usepackage{lmodern}          


 \usepackage[ngerman]{babel}

\begin{document}

\section{2 Conservation laws of fluid motion and boundary conditions}
\begin{itemize}
\item Explain the physical meaning of the different terms in the conservation equations (link between mathematical “operations” and physical behaviour)

Präsentation Folie 7

Impulsgleichung


Energiebilanz

Massenbilanz



\item Explain the physical meaning of the different terms in a general transport equation

dito obne

\item Derive continuity and momentum equations from basic physical idea (mass conservation, Newton second law) and general transport equation with appropriate variable substitutions
Präsentation Folie 5

Kontinuitätsgleichung bedeutet Qellenfrei, Erhaltend

\end{itemize}



\section{3 Turbulence and its modelling}
\begin{itemize}
\item Explain the properties of turbulence and their influence on the flow field. Which changes can be observed compared to laminar flow?

Fole 5
Nichtlinear
Zufälligkeit (nicht Reproduzeirbar)
3D Densional, auch wenn Mittelwert in 1D und 2D variert
hohe Wirbelstärke
Energiewirbel disipiert (wird immer kleiner)
intermitetci, es gibt Nicht turbulenz strömung oder nicht aber auch turbulenz fluss
Hohe diffusitität von Impuls und Energie (gute verteilung)
Trubulenz ist lokaer umwelt abhänig (z.B. Absatz)

\item Explain the main approximations using RANS and LES modelling, including assumptions (what is computed and how, what is modelled and how)

Reynolds- Averaged Navier-Stokes Equations (RANS)\\
Geschwindikeit und Druck wir in einen gemittelten und einen fluchtuierenden Anteil getrennt
%u=u_strich + u'
%p=p_strich + p'
Diese Teile zustälich zu Impulsgleichung
einer neuer Term ein symetrischer Reynoldsspannungstensor von 2ter Sturfe, da es nicht mehr linear ist sonder Multipliziert wir
Folie 14, Trubulenze

Large Eddy Simulation (LES)
Zeitabhänige Simulation
genaue Simulation nicht komerziell


\item What are wall functions, idea behind them, advantages and disadvantages tabelle funktion

grad 0
0
wandfunktion bei hohen reynoldzahl vereinfacht komplexität
dimensionlose Kennzahl der Abstand der ersten Zelle am Rand, siehe Bild Silvio

\end{itemize}


\section{4 The finite volume method for the diffusion problem}
\begin{itemize}

\item Derive a finite-volume discretization for a one-dimensional heat conduction problem. 

mit Unterlagen... Folie 14

\item Demonstrate, using Taylor series expansion, that the central differencing scheme has second order accuracy.

Folie 11, siehe auch Multiphysik


\end{itemize}




\section{5 The finite volume method for convection-diffusion problems}
\begin{itemize}
\item Explain why the CDS (central differencing scheme) does not work for large velocities.
Pe - Klee Zahl verhänihs zeischen Konvektion und Diffusion
Internet

\item Define the Peclet number
Pe - Klee Zahl verhänihs zeischen Konvektion und Diffusion

\item Explain why the upwind discretization (UD) works better. Explain the disadvantages of upwind discretization.
Richtung im Wind, 
disadvatianges:
- es leidet von falscher numerischer
- ist nur 1 ordungng und eshalb eher ungenau
advatianges
- grosse Zeitschritte
- schlechteres netz
- 



\item Explain the terms: conservativeness, boundedness, transportiveness
nachlesen...



\item Explain how the hybrid differencing scheme works.
idea: kombination von upwind und central, 1 Ordnung....
upwind: hoche PeKlee Zahen (hohe konvektion und wenig diffusion)
centrel diffenzing: kleine PeKlee Zahlen (weil mehr diffusion)

\item Explain the QUICK scheme. Advantages and disadvantages.
optimierung von upwind, schaut noch weiter in Zukunft und gewichtet die Zukunft
sehr negativ ist es überschwingt, dies ist total unphysikalisch
positiv näher an der exakten lösung




\item Describe the idea behind TVD schemes.
TVD steht für Total Vaariation Diminishing,
Ideeist die Hügel zu reduzieren
Folie 19



\end{itemize}



\section{6 Solution algorithms for pressure velocity coupling in steady flows}
\begin{itemize} 

\item Difference between incompressible (pressure-based) and compressible (density based) approach/codes => which equations are available for which variable
incompressibel, dichteänderung gleich 0 und p ist konstant
compressible, dichteänderung ...
folen 7


\item Differences between momentum equations and general scalar transport equations => new aspects to be tackled
??? zu folie 11

\item Explain the SIMPLE procedure for pressure-velocity coupling
siehe folie 14

\item Explain the role of relaxation factor. 
Flussdiagramm siehe Folie 14


\item Why are they used?
folie 20


\end{itemize}



\section{7 Solution of discretized equations}
\begin{itemize}
\item Explain why iterative methods are necessary to solve sparse linear systems.
schneller gelöst, numerisch weniger aufwindig
Iterative löser ...
sparse linear system sind einfach pararellisierbar


\item Describe the Jacobi and Gauss-Seidel iterative methods
Jacobi iterative method: 
möglichst nahe an einheitsmatix kommen
eigenwert muss keiner 1 sein, wird dafür langsamer 

Gauss- Seidel method: 
nicht paararellisierbar, konvergit schneller

\item Describe the idea behind the multi-grid method
Anfangsbedingungne 
Stufenweises lösen auf verschiednen netzgrüssen
foleie 19 (Ansys arbeitet so)
feien lösen
grobes netzt lösen
dann wieder feiner...

\end{itemize}



\section{8 The finite volume method for unsteady flows}
\begin{itemize}
\item Describe the three common schemes for time discretization:
\begin{itemize}
\item Explicit (forward Euler)
...

\item Cranck-Nicholson
...

\item Implicit (backward Euler)
...

\end{itemize}
\item What are the advantages and disadvantages of the different schemes?


\item How does the SIMPLE scheme have to be modified for a transient simulation?



\end{itemize}





\section{9 Implementation of boundary conditions}
\begin{itemize}

\item Name 5 important boundary conditions
Dirichlet boundary conditions
Neumann boundary conditions
Robin bandary  condition


inelt
outlet
wall



\item Why cannot symmetry planes (and symmetry boundary conditions always be used in 
CFD)?
\item Why should outlets be placed far away from the interesting flow region?
weil am anfang am rand falsche randbedingungen sind, siehe bild silvio, slide 38


\end{itemize}


\section{10 Errors and uncertainty in CFD modelling}

\begin{itemize}
\item Describe three potential numerical errors in CFD

\begin{itemize}
\item Disketierungsfehler (netzt ist nicht perfekt ist wenn abgeschitten wir
\item Rundungsfehler (flot...)
\item Iterative Konvergenzfehler, Residien???
\end{itemize}
siehe buch 289

\item Describe two typical physical uncertainties in CFD (uncertainties = Unsicherheiten)
limitierte genauigkeit
submodell ist nicht vallierdt
siehe Buch 291

\item What do the terms verification and validation mean.
verification, mathematisch teil, ...
validation, ingenieur, überprüfen von beschreibung der wirklichkeit entspricht


\end{itemize}



Hier eine Mustergleichung... :-) von PartDiff :-)
\[
\frac{\partial^3 u}{\partial x^3}-\frac{\partial u}{\partial y}=0
\]




\end{document}




